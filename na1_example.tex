\documentclass[lang=cn,bibend=bibtex]{elegantpaper}
\title{数值分析作业建议}
\author{Chenchen}
\date{\today}
\usepackage{amsmath}  
\usepackage{amsthm}
% 定义定理样式
\newtheoremstyle{boldexercise} % 名称
{3pt} % 上空间
{3pt} % 下空间
{} % 内容字体
{} % 缩进量
{\bfseries} % 标题字体
{.} % 标题后的标点
{.5em} % 标题后的空间
{} % 定理头部格式

% 应用定理样式
\theoremstyle{boldexercise}
\newtheorem{exercise}{练习}[section]
\newenvironment{solution}{\begin{proof}[\textbf{\emph 解}]}{\end{proof}}
\begin{document}
\maketitle

\section{第一次作业\LaTeX{}排版建议}

\subsection{整体建议}

\begin{enumerate}
\item 建议辰辰可以使用ElegantPaper模板,就像本tex文件一样,这样能提高基础颜值。
\item 建议辰辰可以使用一些如\lstinline|exercise|环境、\lstinline|proof|环境,
  一方面可以自动编号,另一方面可以可以使得tex文件更加整齐,例如:
\end{enumerate}

\begin{exercise}
  这是第一道练习题。
\end{exercise}

\begin{proof}
  这是第一道题的证明。
\end{proof}

\subsection{一些细节建议}

\begin{enumerate}
\item 
  辰辰对有些数学符号可能还不太熟悉,
  例如$\sup$会打成$sup$,$\ln$会打成$ln$,
  不过这些都是小细节问题!
\item 一般数学和叙述并排时长公式尽量使用\lstinline|equation|环境,
  并使用\lstinline|\quad|控制间距,
  例如辰辰的1.4解答可以改为:
  \begin{solution}
    首先有
    \begin{equation}
      \label{eq:1}
      e_{n+1} = x_{n+1} - r = x_n - \frac{f(x_n)}{f^{\prime}(x_n)} - r,
    \end{equation}
    由泰勒展开
    \begin{equation}
      0 = f(r) = f(x_n) + f^{\prime}(\psi_n)(r-x_n),\quad \psi_n \in (r,x_n),
    \end{equation}
    移项得到$f(x_n) = f^{\prime}(\psi_n)(x_n - r)$,将上式代入~(\ref{eq:1})得到
    \begin{equation}
      e_{n+1} = x_n - r - \frac{f^{\prime}(\psi_n)}{f^{\prime}(x_0)}(x_n - r),
    \end{equation}
    即$e_{n+1} = (1 - f^{\prime}(\psi_n)/f^{\prime}(x_0))(x_n - r)$,因此
    \begin{equation}
      C = 1 - \frac{f^{\prime}(\psi_n)}{f^{\prime}(x_0)}, \quad s = 1.\qedhere
    \end{equation}
  \end{solution}
\item 分段函数啥嘟可以问问GPT!
  \begin{equation}
    f(x) =
    \begin{cases}
      a, & x < 1,\\
      b, & x \in [0,1],\\
      c, & x > 1
    \end{cases}
  \end{equation}
\item 尽量少用\lstinline|\\|进行回车,尽量依靠自然换行,
  比如本tex就没有一处使用了\lstinline|\\|,
  辰辰可以试试看不用\lstinline|\\|的换行。
\item 如果一个公式特别长,可以用\lstinline|align|环境,
  例如辰辰1.8中的公式,可以改为:
\end{enumerate}

\begin{align}
  f(x_n)&=f(r)+f'(r)(x_n-r)+\cdots+\frac{f^k(r)}{k!}(x_n-r)^k+\frac{f^{k+1}(\xi)}{(k+1)!}(x_n-r)^{k+1}\\
        &=\frac{f^k(r)}{k!}(x_n-r)^k+\frac{f^{k+1}(\xi)}{(k+1)!}(x_n-r)^{k+1}
\end{align}

\begin{align}
  f'(x_n)&=f'(r)+f''(r)(x_n-r)+\cdots+\frac{f^k(r)}{(k-1)!}(x_n-r)^{k-1}+\frac{f^{k+1}(\psi)}{k!}(x_n-r)^k\\
         &=\frac{f^k(r)}{(k-1)!}(x_n-r)^{k-1}+\frac{f^{k+1}(\psi)}{k!}(x_n-r)^k.
\end{align}

\begin{align}
  x_{n+1}-r&=x_n-r-\frac{f(x_n)}{f'(x_n)}\\
           &=x_n-r-k\cdot\frac{\frac{f^k(r)}{k!}(x_n-r)^k+\frac{f^{k+1}(\xi)}{(k+1)!}(x_n-r)^{k+1}}{\frac{f^k(r)}{(k-1)!}(x_n-r)^{k-1}+\frac{f^{k+1}(\psi)}{k!}(x_n-r)^k}\\
           &=\frac{(\frac{f^{k+1}(\psi)}{k!}-k\cdot\frac{f^{k+1}(\xi)}{(k+1)!})e_n^{k+1}}{\frac{f^k(r)}{(k-1)!}e_{n}^{k-1}+\frac{f^{k+1}(\psi)}{k!}e_{n}^k}
\end{align}



\end{document} 
