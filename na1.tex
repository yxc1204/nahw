\documentclass[lang=cn,bibend=bibtex]{elegantpaper}
\title{数值分析作业}
\author{叶晓辰}
\date{\today}
\usepackage{amsmath}
\usepackage{amsthm}
\newtheoremstyle{boldexercise}
{3pt}
{3pt}
{}
{}
{\bfseries}
{.}
{.5em}
{}

\theoremstyle{boldexercise}
\newtheorem{exercise}{练习}[section]
\newenvironment{solution}{\begin{proof}[\textbf{\emph 解}]}{\end{proof}}
\begin{document}
\maketitle

\section{数值分析第一次作业}
%1.1
\begin{exercise}
  Consider the bisection method starting with the initial interval [1.5,3.5]. In the following questions ``interval''refers to the bisection interval whose width changes across different loops.
\begin{itemize}
  \item What is the width of the interval at the $n$th step?
  \item What is the supremum of the distance between the root $r$ and the midpoint of the interval? 
\end{itemize}
\end{exercise}
\begin{solution}
  (1)第n步的间隔长度:$\frac{1}{2^{n-1}}$.
  (2) $\sup\vert r-c_{n}\vert=\frac{1}{2^{n}}$.
\end{solution}
%1.2
\begin{exercise}
  In using the bisection algorithm with its initial interval as [$a_0$,b$_0$], with $a_0 \textgreater 0$,we want to determine the root with its relative error no greater than $\epsilon$. Prove that this goal of accuracy is guaranteed by the following choice of the number of steps,
  \begin{equation*}
    n \ge \frac{\log(b_0-a_0)-\log\epsilon-\log a_0}{\log 2}-1.
  \end{equation*}
\end{exercise}
\begin{solution}
  由题意
  \begin{equation}
    \label{eq:1.2.1}
    \sup\vert r-c_{n}\vert= \frac{b_{0}-a_{0}}{2^{n+1}}.
  \end{equation}
  相对误差
  \begin{equation}
    \label{eq:1.2.2}
    e=\frac{\sup \vert r-c_{n} \vert}{r}\le \frac{b_{0}-a_{0}}{2^{n+1}\cdot a_{0}}.
  \end{equation}
  由题意有
    $e \textless \epsilon$.
  于是
  \begin{equation}
    \label{eq:1.2.3}
     \frac{b_{0}-a_{0}}{2^{n+1}\cdot a_{0} }\textless \epsilon .
  \end{equation}
  两边同时作用$\ln$,化简即得
  \begin{equation}
    \label{eq:1.2.4}
     n \ge \frac{\ln(b_{0}-a_{0})-\ln\epsilon-\ln a_{0}}{\ln2}-1. 
  \end{equation}
\end{solution}
%1.3
\begin{exercise}
  Perform four iterations of Newton's method for the polynomial equation $p(x)=4x^3-2x^2+3=0$ with the starting point $x_0=-1$. Use a hand calculator and organize results of the iterations in a table.
\end{exercise}
\begin{solution}
  家里计算器不造跑哪去了,不算啦!
\end{solution}
%1.4
\begin{exercise}
  Consider a variation of Newton's method in which only the derivative at $x_0$is used,
  \begin{equation*}
    \label{eq:2}
   x_{n+1}=x_n-\frac{f(x_n)}{f'(x_0)}. 
  \end{equation*}
  Find $C$ and $s$ such that
  \begin{equation*}
    \label{eq:3}
   e_{n+1}=Ce_n^s. 
  \end{equation*}
  where$e_n$ is the error of Newton's method at step n, $s$ is a constant , and $C$ may depend on $x_n$, the true solution $\alpha$, and the derivative of the function $f$.
\end{exercise}
\begin{solution}
  \begin{equation}
    \label{eq:1.4.0}
    e_{n+1}=x_{n+1}-r=x_n-\frac{f(x_n)}{f'(x_0)}-r.
  \end{equation}
  由泰勒展开
  \begin{equation}
    \label{eq:1.4.1}
    0=f(r)=f(x_n)+f'(\psi_n)(r-x_n), \psi_n\in (r,x_n).
  \end{equation}
  移项得到
    $f(x_n)=f'(\psi_n)(x_n-r)$,
  将其代入~(\ref{eq:1.4.0})得
  \begin{equation}
    \label{eq:1.4.4}
    e_{n+1}=x_n-r-\frac{f'(\psi_n)}{f'(x_0)}(x_n-r). 
  \end{equation}
  即
   $ e_{n+1}=(1-\frac{f'(\psi_n)}{f'(x_0)})(x_n-r).$
  所以有
  \begin{equation}
    \label{eq:1.4.6}
    C=1-\frac{f'(\psi_n)}{f'(x_0)}, s=1.
  \end{equation}
\end{solution}
%1.5
\begin{exercise}
 Within$(-\frac{\pi}{2},\frac{\pi}{2})$, will the iteration $x_{n+1}=\arctan x_n$ converge? 
\end{exercise}
\begin{solution}
  \begin{equation}
    \label{eq:1.5.1}
    f(x)=arctanx,  f:(-\frac{\pi}{2},\frac{\pi}{2})\rightarrow (-\frac{\pi}{2},\frac{\pi}{2})
  \end{equation}
  \begin{equation}
    \label{eq:1.5.2}
    \arctan x 
    \begin{cases}
      \ge x,& x \in(\frac{-\pi}{2},0],\\
      \textless x,& x \in (0,\frac{\pi}{2})
    \end{cases}
  \end{equation}
  对$x_0$分类讨论:
  \begin{itemize}
  \item 当$x_0\in(0,\frac{\pi}{2})$时,由归纳知,$\forall n, x_n \in (0,\frac{\pi}{2})$ ,从而$x_{n+1}=arctanx_n\textless x_n$,从而\{$x_n$\}是单调递减的数列。结合其有下界,故由单调有界定理,极限存在,为0.
  \item $x_{0} \in (\frac{-\pi}{2},0]$时,同理可知\{$x_n$\}极限存在,为0.
  \end{itemize}
\end{solution}
%1.6
\begin{exercise}
  Let $p\textgreater1$.What is the value of the following continued fraction?
  \begin{equation*}
   x=\frac{1}{p+\frac{1}{p+\frac{1}{p+\cdots}}} .
  \end{equation*}
  Prove that the sequence of values converges.
\end{exercise}
\begin{solution}
  \begin{equation}
    \label{eq:1.6.1}
    f(x)=\frac{1}{p+x},f:(0,1)\rightarrow (0,1).
  \end{equation}
  \begin{equation}
    \label{eq:1.6.2}
    f'(x)=-\frac{1}{(x+p)^2}, \quad
    \vert f'(x)\vert \le \frac{1}{p^2} \textless 1, \forall x\in(0,1).
  \end{equation}
     \quad\quad 又因为 $x_1=\frac{1}{p}, x_{n+1}=\frac{1}{p+x_n}.$\quad
  由压缩映射的推论易知,\{x$_n$\}收敛.
\end{solution}
%1.7
\begin{exercise}
What happens in problem 1.2. if $a_0\textless0\textlessb_0$? Derive an inequality of the number of steps similar to that in 1.2. In this case,is the relative error still an appropriate measure?
\end{exercise}
\begin{solution}
  由题意
  \begin{equation}
    \label{eq:1.7.1}
    \frac{b_0-a_0}{2^{n+1}\cdot \vert r \vert}\textless \epsilon.
  \end{equation}
  两边同时作用$\ln$,得到
  \begin{equation}
    \label{eq:1.7.2}
    \ln (b_0-a_0)\le(n+1)\ln 2+\ln\vert r \vert +\ln\epsilon.
  \end{equation}
  整理得
  \begin{equation}
    \label{eq:1.7.3}
    n \ge \frac{\ln(b_0-a_0)-\ln\vert r\vert- \ln\epsilon}{\ln2}-1.
  \end{equation}
  因为r未知且可以很靠近0,所以n可能很大,此处控制相对误差不合适。
\end{solution}
%1.8
\begin{exercise}
  Consider solving $f(x)=0$ $(f\in {\cal{C}}^{k+1})$ by Newton's method with the starting point $x_0$ close to a root of multiplicity $k$. Note that $\alpha$ is a zero of multiplicity $k$ of the function $f$ iff
  \begin{equation*}
    \label{eq:4}
    f^{(k)}(\alpha)\neq 0; \quad \forall i \textless k,f^{(i)}(\alpha)=0.
  \end{equation*}
\end{exercise}
\begin{solution}
  (1)此时收敛速度很慢.\\
  (2)
  由泰勒展开\\
  \begin{align}
    f(x_n)&=f(r)+f'(r)(x_n-r)+\cdots+\frac{f^k(r)}{k!}(x_n-r)^k+\frac{f^{k+1}(\xi)}{(k+1)!}(x_n-r)^{k+1}\\
          &=\frac{f^k(r)}{k!}(x_n-r)^k+\frac{f^{k+1}(\xi)}{(k+1)!}(x_n-r)^{k+1}.
  \end{align}
  \begin{align}
    f'(x_n)&=f'(r)+f''(r)(x_n-r)+\cdots+\frac{f^k(r)}{(k-1)!}(x_n-r)^{k-1}+\frac{f^{k+1}(\psi)}{k!}(x_n-r)^k\\
           &=\frac{f^k(r)}{(k-1)!}(x_n-r)^{k-1}+\frac{f^{k+1}(\psi)}{k!}(x_n-r)^k. 
  \end{align}

  将上式代入$x_{n+1}-r&=x_n-r-\frac{f(x_n)}{f'(x_n)}$中
  \begin{align}
    x_{n+1}-r&=x_n-r-\frac{f(x_n)}{f'(x_n)}\\
             &=x_n-r-k\cdot\frac{\frac{f^k(r)}{k!}(x_n-r)^k+\frac{f^{k+1}(\xi)}{(k+1)!}(x_n-r)^{k+1}}{\frac{f^k(r)}{(k-1)!}(x_n-r)^{k-1}+\frac{f^{k+1}(\psi)}{k!}(x_n-r)^k}\\
             &=\frac{(\frac{f^{k+1}(\psi)}{k!}-k\cdot\frac{f^{k+1}(\xi)}{(k+1)!})e_n^{k+1}}{\frac{f^k(r)}{(k-1)!}e_{n}^{k-1}+\frac{f^{k+1}(\psi)}{k!}e_{n}^k}.
  \end{align}

  由上式易知,收敛阶为2.
\end{solution}

\end{document} 