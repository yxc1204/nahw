\documentclass[UTF8]{ctexart}
\title{数值分析作业}
\author{叶晓辰}
\date{\today}
\begin{document}
\maketitle
\section{第一次作业}
\subsection{}
(1)第n步的间隔长度:$\frac{1}{2^{n-1}}$.
(2)$sup\vert r-c_{n}\vert=\frac{1}{2^{n}}$.
\subsection{}
由题意,$sup\vert r-c_{n}\vert= \frac{b_{0}-a_{0}}{2^{n+1}}$.
相对误差:$e=\frac{sup \vert r-c_{n} \vert}{r}\le \frac{b_{0}-a_{0}}{2^{n+1}\cdot a_{0}}$.
由题意有:
  $e \textless \epsilon .$
于是:
\[ \frac{b_{0}-a_{0}}{2^{n+1}\cdot a_{0} }\textless \epsilon .\]
两边同时作用$ln$,化简即得:
\[ n \ge \frac{ln(b_{0}-a_{0})-ln\epsilon-lna_{0}}{ln2}-1. \]
\subsection{} 
家里计算器不造跑哪去了,不算啦!
\subsection{}
$e_{n+1}=x_{n+1}-r=x_n-\frac{f(x_n)}{f'(x_0)}-r.$ \\
由泰勒展开,$0=f(r)=f(x_n)+f'(\psi_n)(r-x_n), \psi_n\in (r,x_n)$.\\
移项得到,$f(x_n)=f'(\psi_n)(x_n-r)$.\\
将上式代入(1),$e_{n+1}=x_n-r-\frac{f'(\psi_n)}{f'(x_0)}(x_n-r)$. \\ 
即,$e_{n+1}=(1-\frac{f'(\psi_n)}{f'(x_0)})(x_n-r).$\\
所以有,$C=1-\frac{f'(\psi_n)}{f'(x_0)}, s=1.$
\subsection{}
$x_{n+1}=arctanx_n$\\
$f(x)=arctanx,  f:(\frac{-\pi}{2},\frac{\pi}{2})\rightarrow (\frac{-\pi}{2},\frac{\pi}{2})$
\begin{itemize}
\item 当$x \in (\frac{-\pi}{2},0]$时,$arctanx \ge x$,
\item 当$x \in (0,\frac{\pi}{2})$时,$arctanx \textless x$.
\end{itemize}
(这里的分段函数死活打不出来qaq)\\
对$x_0$分类讨论:
\begin{itemize}
\item 当$x_0\in(0,\frac{\pi}{2})$时,由归纳知,$\forall n, x_n \in (0,\frac{\pi}{2})$ ,从而$x_{n+1}=arctanx_n\textless x_n$,从而\{$x_n$\}是单调递减的数列。结合其有下界,故由单调有界定理,极限存在,为0.
\item $x_{0} \in (\frac{-\pi}{2},0]$时,同理可知\{$x_n$\}极限存在,为0.
\end{itemize}
\subsection{}
$x_1=\frac{1}{p}, x_{n+1}=\frac{1}{p+x_n}$.\\
$f(x)=\frac{1}{p+x},f:(0,1)\rightarrow (0,1),f'(x)=-\frac{1}{(x+p)^2}.$\\
$\vert f'(x)\vert \le \frac{1}{p^2} \textless 1, \forall x\in(0,1)$.\\
由压缩映射的推论易知,\{x$_n$\}收敛.
\subsection{}
此时,$\frac{b_0-a_0}{2^{n+1}\cdot \vert r \vert}\textless \epsilon$.\\
两边同时作用$ln$,得到,$ln(b_0-a_0)\le(n+1)ln2+ln\vert r \vert +ln\epsilon$.\\
整理得,$n \ge \frac{ln(b_0-a_0)-ln\vert r\vert- ln\epsilon}{ln2}-1$.
因为r未知且可以很靠近0,所以n可能很大,此处控制相对误差不合适。
\subsection{}
(1)此时收敛速度很慢.\\
(2)
由泰勒展开\\
$f(x_n)=f(r)+f'(r)(x_n-r)+\cdots+\frac{f^k(r)}{k!}(x_n-r)^k+\frac{f^{k+1}(\xi)}{(k+1)!}(x_n-r)^{k+1}=\frac{f^k(r)}{k!}(x_n-r)^k+\frac{f^{k+1}(\xi)}{(k+1)!}(x_n-r)^{k+1}.$\\
$f'(x_n)=f'(r)+f''(r)(x_n-r)+\cdots+\frac{f^k(r)}{(k-1)!}(x_n-r)^{k-1}+\frac{f^{k+1}(\psi)}{k!}(x_n-r)^k=\frac{f^k(r)}{(k-1)!}(x_n-r)^{k-1}+\frac{f^{k+1}(\psi)}{k!}(x_n-r)^k.$\\ 
$x_{n+1}-r=x_n-r-\frac{f(x_n)}{f'(x_n)}=x_n-r-k\cdot\frac{\frac{f^k(r)}{k!}(x_n-r)^k+\frac{f^{k+1}(\xi)}{(k+1)!}(x_n-r)^{k+1}}{\frac{f^k(r)}{(k-1)!}(x_n-r)^{k-1}+\frac{f^{k+1}(\psi)}{k!}(x_n-r)^k}=\frac{(\frac{f^{k+1}(\psi)}{k!}-k\cdot\frac{f^{k+1}(\xi)}{(k+1)!})e_n^{k+1}}{\frac{f^k(r)}{(k-1)!}e_{n}^{k-1}+\frac{f^{k+1}(\psi)}{k!}e_{n}^k}$.\\
由上式易知,收敛阶为2.


\end{document} 